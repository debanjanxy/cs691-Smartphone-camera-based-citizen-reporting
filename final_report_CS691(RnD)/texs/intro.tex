\chapter{Introduction}
Two-wheeler vehicles are common means of transport in India. The major reason is its low price and ease of operation as compare to other vehicles. But safety is a matter of high concern when it comes to two-wheeler vehicles. Observing the importance of safety for two-wheeler vehicles, governments have made it a punishable offense to break traffic rules like driving without helmet and others like using mobile phones while driving. Therefore it motivates us to build a system that can help in reporting against this two-wheeler traffic rules violation. For collection of traffic violation images, we have made an Android app which can record traffic rules violation incidents. Generally, the moment at which we recognize a traffic rule violation, the incident has already happened. Thus we have introduced this facility in our app to capture past 20 seconds of video whenever we touch the screen while the app is opened.  After collection of traffic rules violation videos, we have decomposed them into frames. On each frame, we have applied our approach for detecting traffic rules violation. We have proposed an approach to automatically recognize motorcycle riders and passengers whether they are wearing helmets or not using various feature extraction and classification techniques. These classification techniques will output only those images where the bike rider is not wearing the helmet. Then these images will be fed to a different model that will try to detect it’s license number from its license number plate.
\par Main challenge in this task is that there are no standard datasets for helmet and non-helmet vehicles. Similarly collecting a dataset of Indian number plates is also challenging. In the paper \cite{b4} and \cite{b5} it has been suggested to use sophisticated techniques like a histogram of oriented gradients (HOG), scale-invariant feature transform (SIFT), and local binary patterns (LBP). We decided to go with different convolutional neural network (CNN) architectures for feature extraction. And for classification, we have used various classifiers like multi-layered perceptron (MLP), support vector machines (SVM), etc.
\par For number plate detection we have used an open source tool called OpenALPR \cite{b0} which can detect and recognize the number plates of the vehicles present in an image. We have used the tool using online API calls to its web server. Apart from this technique, we have also implemented another method which can be done in three steps. They are i) number plate localization, ii) character segmentation and iii) character recognition. After recognizing the number plate successfully, we can report the vehicle number to the traffic police department for proper actions.
\par Our solution aims to spread awareness in the people and make them feel the responsibility of creating their surrounding world a better place.
